\documentclass[../../main.tex]{subfiles}

%% Preamble
\usepackage{mathtools}


\begin{document}

Symbols in strings, for example, \( a, b \), always belong to an alphabet.
Hence, we can say an alphabet \(\sum\) is a set of characters, for example \(\{a, b\}\), is formally defined as:

\begin{equation} \label{alphabet}
Alphabet = \sum = \big\{ a, b \big\} 
\end{equation}

A language, \( L \), is a set over a particular alphabet. Hence:

\begin{equation} \label{language}
    Language = L1(\Sigma) = \big\{ a, aa, b, ab, ba, bba, ... \big\} 
\end{equation}

As you may see, the above language \( L1 \) is generated over the alphabet \( \Sigma \), and contains various combinations of the symbols in our alphabet. In theory,
it is possible to create an infinite number of such combinations. Therefore, that language is an \textit{infinite} language. \\

As you might have guessed, there are two main types of languages:

\begin{itemize}
\item infinite
\item finite
\end{itemize}

For a language to be finite, we must restrict how symbols are combined. Such restrictions applied to a language produce a grammar, \( G \).
Formally, we say that a grammar \( G \) is a tuple:

\begin{equation} \label{grammar}
    Grammar = G = ( N, T, P, S )
\end{equation}

where \(N\) = nonterminals, \(T\) = terminals, \(P\) = produtions, or rules, and \(S\) = starting symbol.


\end{document}